%*******************************************************
% Abstract
%*******************************************************

\chapter*{Abstract}\label{chp:abstract}

The increased use of technology in modern vehicles has made cybersecurity a crucial part of the development process of modern vehicles,
making it increasingly critical for the safety and security of the vehicle as well as the privacy and personal data of drivers and passengers.\\

The internal vehicular network of such E/E systems plays a vital role in the overall cybersecurity of a modern vehicle.
For example, attackers might exploit potential attack paths in the network to gain unauthorized access to a vehicle's systems or networks.
Evaluation of the security of vehicular network architectures is a crucial step in the development process of modern vehicles,
however it is a challenging task due to their increased complexity and the large number of possible attack paths.\\

This thesis focuses on finding a criteria to evaluate the security of vehicular network architectures by
comparing the security of different vehicular network architectures based on their attack path feasbility.
We conducted a survey with security experts to rank a test set of vehicular network architectures based on their security.
Using the results of a survey, we calibrated different criteria to evaluate the security of vehicular network architectures. 
We then applied the criteria on a proof-of-concept set of vehicular network architectures to evaluate their security and conclude whether the criteria were properly calibrated.\\

Our results show that the calibration was a success and representing the criteria as a mathematical equation is a 
viable approach to evaluate the security of vehicular network architectures in an automated and efficient manner.