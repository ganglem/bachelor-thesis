\chapter{Deciding on the criteria}
\label{chp:criteria}

One of the most important asepcts of this thesis is the criteria used to evaluate the different architectures.

In order to find and calibrate a good criteria, a survey was conducted in which ten architectures were evaluated by security experts in the field of automotive security.
Their rating and feedback on this "training set" was used to calibrate the criteria used on the main architectures in this thesis.
Doing a survey with such a set rather than the architectures themselves was done to avoid biasing the results, as well as have a well evaluated criteria.
In addition, the training set is much larger than the main set, thus the results are more accurate.

\section{Survey}
\label{sec:survey}

Security experts at Mercedes-Benz Tech Innovation were given ten architectures as shown in \ref{chp:arch}.
The experts were asked to rank the architectures and give reasoning behind their ranking, as well as add any comments they had.
Since there are ten architectures, the ranking went from 1 to 10, where 1 is the most secure and 10 is the least secure.
To get to a result in the end, each rank was given a score, where 1 got 1 points, 2 got 2 points, and so on.
In the end, each ranking or score each architecture got was summed up, and the architecture with the lowest score was the most secure.

The result is as follows:

\begin{table}[h]
    \label{table:survey}
    \centering
    \caption{Rank and Architecture}
    \begin{tabular}{ |c|c| } 
    \hline
    Rank & Architecture \\
    \hline
    1 & \ref{fig:architecture3}\\
    2 & \ref{fig:architecture8}\\
    3 & \ref{fig:architecture6}\\
    4 & \ref{fig:architecture10}\\
    5 & \ref{fig:architecture2}\\
    6 & \ref{fig:architecture1}\\
    7 & \ref{fig:architecture5}\\
    8 & \ref{fig:architecture7}\\
    9 & \ref{fig:architecture9}\\
    10 & \ref{fig:architecture4}\\
    \hline
    \end{tabular}
\end{table}

Of course there are many ways how a security architect approaches an evaluation.
Each architect comes from a different background with different experience and thus will see some factors as more or less important than another.
Even though architects tend to agree in general, it's in the details where opinions will digress.
This result thus represents the architects' mean opinion on the security of the architectures. 

The experts agreed that the number of hops between the entry and target is a critical criterion in evaluating the security of attack paths. 
The more hops there are, the more challenging it becomes for attackers to navigate from the entry point to the target, however not every hop is the same.\\

Separation and isolation, both of domains and ECUs, is a crucial consideration, as it enables compartmentalization and application of security measures to each domain or ECU individually. 
The Powertrain domain, which controls the engine, transmission, and other critical systems, is of particular concern, and its position in the network must be carefully considered. 
It was thus somewhat expected that Architecture \ref{fig:architecture3} received the highest score, as it has all of the external intefaces placed in the same domain.
Architecture \ref{fig:architecture10} was also expected to receive a high score, as it is similar to Architecture 3 \ref{fig:architecture3} in terms of the placement of the external interfaces,
but the Telematic domain controller being a target was a concern.
Architecture \ref{fig:architecture6} was also not expected to receive a high score, as the the maximum number of hops between the entry and target is 2 on average,
but the experts agreed that isolation played a huge role in the evaluation.\\

However, excessive isolation would make the system too complex. 
Architectures \ref{fig:architecture6} and \ref{fig:architecture7} also provide high levels of isolation, but at the cost of increased overall complexity.
Communication between ECUs must also be considered, as it is an essential part of the vehicle. 
Too many isolated ECUs communicating with each other can result in significant overhead. 
Media change, such as from CAN to CANFD or FlexRay or Ethernet, also introduces additional hurdles.\\

Other factors that need to be evaluated include the presence of a Central Gateway and the number of external interfaces, as more interfaces mean more attack vectors and the need for more security mechanisms.
Architecture \ref{fig:architecture4} and Architecture \ref{fig:architecture9} have most spread out interfaces, and thus they were expected to receive a low score,
whereas Architectures \ref{fig:architecture8} was expected to receive a high score, as the Central Gateway is the only entry vector into the network.
However, a Central Gateway with no external interfaces is overall less secure than an architecture with no Central Gateway at all, and that is why 
Architecture \ref{fig:architecture2} received a higher score than Architecture \ref{fig:architecture1}. 
Of course, the type and the security of the external interface plays a role as well, because every interface has different security properties.\\

While the attack feasibility of each component accounts for the security mechanisms implemented in the vehicle, such as firewalls or IDCs, 
these mechanisms and the component's feasibility have not been explicitly stated in the architectures under consideration, since every component can vary in their implementation.
However, these feasibility assessments are represented through the feasibility of each ECU, bus, and interface as explained in Section \ref{sec:config}.
The experts agreed that the presence of security mechanisms is a critical factor in evaluating the security of the architectures.
Such security measures impede attackers from progressing quickly through the attack path. 
For example, intelligent domain controllers that differentiate between domains are more effective than those that only forward messages.
As a result, the final results varied between experts, as expected.\\

In general, Architectures \ref{fig:architecture3}, \ref{fig:architecture6}, \ref{fig:architecture8}, and \ref{fig:architecture10} received a positive feedback,
whereas Architectures \ref{fig:architecture4}, \ref{fig:architecture7}, and \ref{fig:architecture9} received a negative feedback.

\section{Criteria}

Just as there are many ways to evaluate an architecture, there are many ways to a general criteria for this thesis.
Similar to \cite{threat_surf}, we decided to approach the criteria as a mathematical equation that can be applied to every architecture equally.
This equation takes the different aforementioned factors into account:

\begin{enumerate}
    \item Architecture feasibility
    \item Attack Path feasibility
    \item Presence and influence of a Central Gateway
    \item Amount of hops 
    \item Isolation of ECUs
    \item Amount of interfaces/entry points
\end{enumerate}

Finding the most fitting equation was done in a manual, brute-force like attempt.

Iterating over each of the ten architectures, their factors were given a weight and were then arranged in different mathematical operations.
These weights were iterated from factors between 0 and 100, resulting in millions of combinations for every attempted equation.
Each combination of the ten architectures using the same weights was then represented as a ranked table. 
In the end, the weights of the tables which had the smallest eucledian distance to the survey table \ref{table:survey} were taken into consideration.
There were hundreds of feasible weight combinations, and in the end the smallest possible weights were chosen.

Finding an equation that fit the criteria was a step-by-step progress.
In genreal, we asked ourselves the questions:\\
\textit{Which of the factors is it favorable to have more of?} and \textit{Which of the factors is it unfavorable to have more of?}

The resulting equation is as follows and is explained below:

Let's consider any one architecture:\\
It is clear that a high architecture feasibility score and a high number of hops in each attack path is the most desirable.
To get the architecture feasibility, each attack path feasibility is multiplied by the amount of hops it has.
 
Since the Bellman-Ford algorithm finds and takes the lowest weighted attack path, i.e. the least secure one, 
a high score for this path means the most feasibile path receives a high security rating.
It is also taken into account how many hops there are. \\

\begin{equation}
    \mathit{A} := {Architectures} a \in \mathit{A} \label{eq:architectures}\\\\
\end{equation}

\begin{equation}
    \mathit{P} := {Attack Paths_{a}} p \in \mathit{P} \label{eq:attack_paths}\\\\
\end{equation}

\begin{equation}
    feasibility_{a} := \sum_{p} feasibility_{p} * hops_{p} * cgw_{a} \label{eq:feasibility}\\\\
\end{equation}

\begin{equation}
    cgw_{a} := 
    \begin{cases}
    1,&\text{base case}\\
    cgw_{a} - 0.2,&\text{if } \not\in a\\\\
    cgw_{a} - 0.3,&\text{if external interface} \in a\\\\
    cgw_{a} - 0.1,&\text{if target} \in a\\\\
    \end{cases} \label{eq:cgw}\\\\
\end{equation}

\begin{equation}
    hops_{a} := \sum_{p} hops_{p} \label{eq:hops}\\\\
\end{equation}

\begin{equation}
    interfaces := \text{total amount of entry points in } a \label{eq:interfaces}\\\\
\end{equation}

\begin{equation}
    separation := \text{average amount of ECUs per bus in } a \label{eq:separation}\\\\
\end{equation}

\begin{equation}
    rating_{a} := \frac{100 * feasibility_{a} * cgw_{a}}{hops_{a}*w_{1} + isolation_{a}^{w_{2}} + interfaces_{a}*w_{3}} \label{eq:rating}\\\\
\end{equation}


