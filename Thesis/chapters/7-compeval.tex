\chapter{Proof of Concept}
\label{chp:compeval}

To prove that the criteria are properly calibrated and gives desired results, another set of architectures is evaluated.
These architectures were designed with the survey feedback in mind and they are referred to as \textit{proof of concept architectures}.
The training set architectures sometimes differ immensely from each other to be able to better crystallize the criteria,
whereas these now are kept relatively similar to each other to ensure the criteria are not biased towards a specific architecture.
Similarly to the test set, this chapter will describe the architectures, estimate the desired results, and evaluate the architectures using the tool.
The following architectures all use the same building blocks with the same values as described in (\ref{chp:arch}).

\section{Architecture A}
\label{sec:archa}

\nameref{fig:architectureA} features four entry points and six interfaces - the highest amount of all the architectures.
However, the strong point here is that they are relatively separated from the rest of the components.
In addition, it features a \acrshort{cgw} that is not an entry point.
These factors indicate a high amount of hops for some targets.
In terms of isolation of \acrshort{ecu}s, most \acrshort{ecu}s share their bus with at least two or 
more other \acrshort{ecu}s, resulting in low isolation compared to the other architectures.
Based on the criteria, this architecture is expected to do the best out of the three architectures.

\section{Architecture B}
\label{sec:archb}

\nameref{fig:architectureB} features three entry points and five interfaces.
Though the number is lower than that of \nameref{fig:architectureA}, the entry points are more spread out, meaning that the maximum amount of hops is lower.
The \acrshort{cgw}, however, is not an entry point, and the isolation of \acrshort{ecu}s is better compared to \nameref{fig:architectureA}.
This architecture, though offering better isolation, is expected to perform worse than \nameref{fig:architectureA} due to the lower amount of hops.

\section{Architecture C}
\label{sec:archc}

\nameref{fig:architectureC} has only three entry points and three interfaces - however, one of them being the \acrshort{cgw} and 
in addition, the entry points are distributed across the architecture. 
This results in a lower amount of hops for the targets, putting this architecture at a disadvantage compared to the other two.
Isolation of \acrshort{ecu}s here is similar or somewhat better to \nameref{fig:architectureB}.
Due to the low amount of hops and the \acrshort{cgw} being an entry point, this architecture is expected to perform 
similar to \nameref{fig:architectureB}, but still the worst out of the three.


\section{Results - Proof Of Concept Architectures}
\label{sec:resultsproof}

The results are as follows:\\

\begin{table}[h]
    \label{table:pocranking}
    \centering
    \begin{tabular}{ |c|c|c| } 
    \hline
    Rank & Architecture & Rating\\
    \hline
    1 & \nameref{fig:architectureA} & 152.97\\ 
    2 & \nameref{fig:architectureB} & 129.2\\
    3 & \nameref{fig:architectureC} & 124.66\\
    \hline
    \end{tabular}
    \caption{Proof of concept ranking \textit{(detailed results of the proof of concept can be found in the appendix (\ref{app:proofOfConcept}))}}
\end{table}

\hfill \break

Taking the criteria into account, the results reflect the expectations.
\nameref{fig:architectureA}, with the highest amount of hops, due to the secluded entry points, is ranked first.
\nameref{fig:architectureB}, with a lower amount of hops but better isolation, is ranked second.
\nameref{fig:architectureC}, with the lowest amount of hops and the \acrshort{cgw} being an entry point, is ranked third.\\

Architecture A was expected to perform the best due to the high amount of hops and the secluded entry points.
It was predicted that \nameref{fig:architectureB} and \nameref{fig:architectureC} would perform similarly and score lower than \nameref{fig:architectureA},
but the difference between \nameref{fig:architectureB} and \nameref{fig:architectureC} is smaller than initially thought. 
However, we expected \nameref{fig:architectureB} to perform better than \nameref{fig:architectureC}, 
because the \acrshort{cgw} being an entry point in \nameref{fig:architectureC} is the deciding factor for the lower score.\\

Overall, we are satisfied with the results, as our predictions matched the actual results.
In conclusion, this proof of concept confirms that the criteria are properly calibrated and give the desired results.