\chapter{Tool}
\label{chp:tool}

In order to evaluate the different complex architecture, a tool was developed.
The tool's purpose is to help security architects quickly evaluate the given vehicular network architecture based on their \gls{attack path} feasibility.\\
Furthermore, the tool is used in this thesis to automatically generate the most feasible attack paths for the different complex architectures, 
as well as evalaute them based on the criteria.

The following sections will describe the tool's implementation.

\section{Configuration Files Structure}
\label{sec:config}

TODO: decide on the numbers

First, the architecture, ECUs and buses need to be defined in a configuration file.
The configuration files for the simulation are stored in JSON format, which is a lightweight data interchange format that is easyto read and write, and easy for machines to parse and generate.

There are three main configuration files used in the simulation: buses.json, ecus.json, and graph.json.
Each file has a specific structure and contains different attributes.

Note that the feasibility rating is between 0 and 1, where 0 is most feasible, i.e. the attack is most likely to succeed, and 1 is least feasible, i.e. the attack is least likely to succeed.

\subsection{buses.json}
\label{sec:buses}

The tool takes as input three sets of configuration files, each of them represented in the JSON format: 
the system architecture, the ECUs in the system, and the buses connecting the ECUs. 

This file defines the different communication buses that are used in the simulation. 
It contains an array of objects, where each object represents a bus and has the following attributes:

\begin{itemize}
\item \textbf{type}: The type of the bus: \textit{CAN, CANFD, FlexRay, Ethernet, MOST, LIN}.
\item \textbf{feasibility}: A value between 0 and 1 indicating the bus' attack feasibility.
\end{itemize}

\subsection{ecus.json}
\label{sec:ecus}

This file defines the electronic control units (ECUs) that are used in the simulation. 
Note that external interfaces (e.g., Bluetooth, WiFi, GPS) are also considered as ECUs in the simulation.
Treating them as ECUs simplifies the implementation of the tool, as not the ECU is treated as entry but rather the technology of the interface.
It contains an array of objects, where each object represents an ECU and has the following attributes

\begin{itemize}
\item \textbf{name}: The name of the ECU.
\item \textbf{type}: The type of the ECU (e.g., entry, target, interface).
\item \textbf{feasibility}: A value between 0 and 1 indicating the ECU's attack feasibility.
\end{itemize}

\subsection{graph.json}
\label{sec:graph}

This file defines the topology of the system being simulated, i.e. the architecture. 
It contains an array of objects, where each object represents a communication link which contains the ECUs and interfaces and has the following attributes:

\begin{itemize}
\item \textbf{name}: The name of the link.
\item \textbf{type}: The type of the link (\textit{CAN, CANFD, FlexRay, Ethernet, MOST, LIN, Bluetooth, WiFi, Ethernet, GNSS}).
\item \textbf{ecus}: An array of the names of the ECUs that are connected by this link.
\end{itemize}


\section{Tool Implementation}
\label{sec:implementation}

The tool is implemented in Python 3.10 and newer.
Python was chosen as the implementation language because it 