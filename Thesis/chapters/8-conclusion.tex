\chapter{Conclusion}
\label{chp:conclusion}

In this thesis we have explored the security of automotive network architectures.\\\par

We have done this by first researching the current state of the art in automotive network security,
finding that the current state of the art is lacking in the area of security evaluation.
To fill this gap, we have developed a new method for evaluating the security of automotive network architectures.
We conducted a survey to find out what factors are important for security architects when evaluating an architecture,
and used these factors to develop a criteria that can be used to evaluate the security of an architecture.
A tool helped us to automate the evaluation of architectures, by automatically generating model-based architectures,
finding their most feasibile attack path and calculating their security feasibility by applying the criteria.\\

Key findings showed that the most important factors for security architects are the attack path feasibility,
hops between entry points and targets, isolation of \gls{ecu} on a bus, and the presence and influence of a \gls{cgw},
as well as the amount and distribution of entry points actross a network architecture.\\

The results and findings of this thesis can be used to aid security architects in their evaluation 
and conception of future automotive network architectures by also providing a 
tool that automoizes the process of graph generation, attack path finding, evaluation, and comparison of architectures.\\\par

Despite providing valuable insights into the security of automotive network architectures,
this thesis is not without its limitations.\\

The criteria is based on the opinions of a small pool of security experts.
More security experts should be consulted to further validate the criteria.\\
In addition, the architectures in the training set as well as the training set itself used for the survey was small.
A larger training set would have provided more data to work with, and would have allowed for a more accurate criteria.
More accurate and architectures actually used in the automotive industry would have also been beneficial, as out 
architectures were kept small and simple.
This was done to keep the conception and survey short and simple, but it would have been interesting to see how the criteria would have
performed on more complex architectures.\\
Another limitation was the way the criteria was developed and calibrated.
Using machine learning to build a neural network that would have learned the criteria would have 
been more accurate and would have allowed for more accurate criteria.\\

Based on the findings and limitations, future work could include the aforementioned suggestions for improvement.
In addition, the work of a security architect for this process could be further automated by 
having different data formats for the input files, such as XML, PDF, or image files, and generation of a report on the security of the architecture.\\

Despite these limitations, our research has shown to be successful as the criteria, represented as a mathematical formula, 
was able to closely replicate the evaluaiton of the security experts.\\

Overall, this thesis has provided valuable understanding on how can different E/E architectures be rated based on their attack path feasibility,
what architectural approach makes a network safer than others and how secure a vehicular network architecture is by using a criteria 
calibrated by security experts.\\