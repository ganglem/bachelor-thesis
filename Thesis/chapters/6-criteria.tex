\chapter{Criteria}
\label{chp:criteria}

Just as there are many ways to evaluate an architecture, there are many ways to a general criteria for this thesis.
Similar to \cite{threat_surf}, we decided to approach the criteria as a mathematical equation that can be applied to every architecture equally.
This equation takes into account the architecture feasibility, attack path feasibility, presence and influence of a Central Gateway,
amount of hops, isolation of ECUs and amount of interfaces/entry points.\\

Finding the most fitting equation was done in a manual, brute-force like attempt.\\\par

We define a set of architectures $\mathit{A}$, where each architecture $a$ has a set of attack paths $\mathit{P}$:\\
\begin{equation}
    \mathit{A} := {Architectures} \tab a \in \mathit{A} \label{eq:architectures}
\end{equation}
\begin{equation}
    \mathit{P} := {Attack Paths_{a}} \tab p \in \mathit{P} \label{eq:attack_paths}
\end{equation}

\hfill \break

Next, we define the factors that are taken into account for the equation.\\

Since the Bellman-Ford algorithm finds and takes the lowest weighted attack path, i.e. the least secure one, 
a high score for this path means the most feasibile path receives a high security rating.
Multiplying the feasibility of each attack path with the amount of hops it has, we also take into account the amount of hops in each attack path.
The more hops there are, the better.
This is because the more hops there are, the more ECUs there are to compromise, 
and the more ECUs there are to compromise, the more likely it is that the attacker will be hindered to reach the target ECU.\\
We calculate the architecture feasibility $feasibility_{a}$ by multiplying the feasibility of each attack path $feasibility_{p}$ with the amount of hops $hops_{p}$ in each attack path:
\begin{equation}
    feasibility_{a} := \sum_{p} feasibility_{p} * hops_{p} * cgw_{a} \label{eq:feasibility}
\end{equation}

\hfill \break

The Central Gateway also plays a role in the security of an architecture.
As mentioned before, the benefit of a Central Gateway is dependent on whether it is present, an entry point, and or a target.
A Central Gateway is beneficial if it is present, but not an entry point or a target.
The Central Gateway factor is 1 by default, and is then decreased by 0.15 if the Central Gateway is not present,
by 0.1 if the Central Gateway is an entry point, and by 0.05 if the Central Gateway is a target.
This reflects the opinion of the Security Architects.\\
To include the influence of a Central Gateway, we define the Central Gateway factor $cgw_{a}$ as follows:\\
\begin{equation}
    cgw_{a} := 
    \begin{cases}
    1,&\text{base case}\\
    cgw_{a} - 0.15,&\text{if } \not\in a\\
    cgw_{a} - 0.1,&\text{if external interface} \in a\\
    cgw_{a} - 0.05,&\text{if target} \in a\\
    \end{cases} \label{eq:cgw}
\end{equation}

\hfill \break

To later norm the amount of hops, we set the hops factor $hops_{a}$ as follows:\\
\begin{equation}
    hops_{a} := \sum_{p} hops_{p} \label{eq:hops}
\end{equation}

\hfill \break

Isolation of an architecture is defined as the amount of ECUs per bus, which we call the separation factor $separation_{a}$:\\
\begin{equation}
    isolation_{a} := \text{average amount of ECUs per bus in } a \label{eq:isolation}
\end{equation}

\hfill \break

And lastly, we define the amount of entry points as the interfaces factor $interfaces_{a}$:\\
\begin{equation}
    entrypoints_{a} := \text{total amount of entry points in } a \label{eq:interfaces}
\end{equation}

\hfill  \break

Iterating over each of the ten architectures, their factors were given a weight and were then arranged in different mathematical operations.
These weights were iterated from factors between 0 and 100, resulting in millions of combinations for every attempted equation.
Each combination of the ten architectures using the same weights was then represented as a ranked table. 
In the end, the weights of the tables which had the smallest eucledian distance to the survey table \ref{table:survey} were taken into consideration.
There were hundreds of feasible weight combinations, and in the end the smallest possible weights were chosen.

Finding an equation that fit the criteria was a step-by-step progress.
In genreal, we asked ourselves the questions:\\
\textit{Which of the factors is it favorable to have more of?} and \textit{Which of the factors is it unfavorable to have more of?}\\\\

Let's consider any one architecture:\\
It is clear that a high architecture feasibility score and a high number of hops in each attack path is the most desirable.
To get the architecture feasibility, each attack path feasibility is multiplied by the amount of hops it has \ref{eq:feasibility}.
We know that the more difficult the path, the more secure the architecture and the more hops there are in a path, the more difficult it is to reach the target ECU.\\

This is then muplitplied by the Central Gateway factor \ref{eq:cgw}.\\

To get a reasonable rating score, the product of the feasibility and Central Gateway factor are multiplied by 100.\\


Next, we consider the factors that are unfavorable to have more of.\\

The isolation of an architecture, i.e. the amount of ECUs per bus, is unfavorable to have more of.
This is because the more ECUs there are per bus, the more ECUs there are to compromise on one bus.
Taking a look back at \nameref{fig:architecture6}, which received a good rating, we can deduce that the more isolated the ECUs are, the better.\\

The amount of interfaces is also unfavorable to have more of.
In this case, we only take into account the amount of entry points, and not the amount of total interfaces.
Doing so, we are able to differentiate between one ECU with many interfaces and many ECUs with one interface since more entry points are also likely more spread out.
Combining entry points and the amount of hops, we can deduce whether the entry points are secluded or spread out.
High hops most likely means that the entry points are all grouped together, and low hops means that the entry points are spread out.\\\\

In the end, we receive the following equation:\\
\begin{equation}
    rating_{a} := \frac{100 * feasibility_{a} * cgw_{a}}{hops_{a}*w_{1} + isolation_{a}^{w_{2}} + interfaces_{a}*w_{3}} \label{eq:rating}\\\\
\end{equation}

\hfill \break