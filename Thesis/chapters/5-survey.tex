\chapter{Criteria preparation}
\label{chp:introcriteria}

One of the most critical aspects of this thesis is the criteria used to evaluate the different architectures.

In order to find and calibrate criteria, a survey was conducted in which security experts in the field of automotive security evaluated ten architectures.
The criteria were calibrated based on their ratings and feedback on this "training set".
Surveying with such a set rather than the architectures was done to avoid biased results and have well-evaluated criteria.

\section{Training set architectures}
\label{sec:trainingarch}

The training set comprised ten distinct architectures, which served to define and calibrate the evaluation criteria. 
The following section will delve into the test architectures, 
providing a description and hypothesizing the impact of architectural choices on system security.

\section{Architecture 1}
\label{subsec:arch1}

\nameref{fig:architecture1} is a simplified rendition of an architecture applied in a real vehicle. 
It presents six entry points and eight targets, including three that involve the \acrshort{cgw}. 
The multitude of entry points amplifies the attack surface and diversifies the attack paths. 
We anticipate this architecture may underperform in terms of security compared to the others in this set due to its extended attack surface.

\section{Architecture 2}
\label{subsec:arch2}

\nameref{fig:architecture2} is essentially the same as \nameref{fig:architecture1}, 
but without a \acrshort{cgw}.
This setup allows us to analyze how the architecture would function without it. 
We predict that this architecture's performance will align closely with or perhaps 
fall slightly short of \nameref{fig:architecture1} and \ref{fig:architecture5}.

\section{Architecture 3}
\label{subsec:arch3}

\nameref{fig:architecture3} was designed to group all the entry interfaces to see how a centralized entry would affect the architecture. 
By confining potential entry locations, the attack surface is reduced, the attack path to each target is elongated, 
and more \acrshort{ecu}s are utilized in the path, making an attack more challenging as the intruder would need to compromise more \acrshort{ecu}s. 
This architecture, similar to Architecture 10, also centralizes entry points but does not confine them to a single \acrshort{ecu}.
We anticipate this architecture to be among the most secure in this set due to its confined entry points.
\section{Architecture 4}
\label{subsec:arch4}

\nameref{fig:architecture4} offers entry points from all domain controllers, each of which has only one type of interface. 
The inclusion of an entry point from each domain controller expands the attack surface. 
To gain quick access to the whole domain, an attacker would only need to compromise one \acrshort{ecu} within each domain controller. 
This architecture, however, is essentially equivalent to \nameref{fig:architecture9}, including the \acrshort{cgw}. 
We anticipate that due to its numerous and dispersed entry points, this architecture might be one of the least secure within this set.
\section{Architecture 5}
\label{subsec:arch5}

\nameref{fig:architecture5} uses only Ethernet as the bus system. 
The purpose here is to examine how the architecture would perform without incorporating any other bus system. 
Ethernet was selected due to its highest security feasibility rating among bus systems. 
Moreover, its rising popularity in the automotive domain makes it a valuable insight.
We expect this architecture to perform similarly to \nameref{fig:architecture1} and \nameref{fig:architecture2}.

\section{Architecture 6}
\label{subsec:arch6}

\nameref{fig:architecture6} maps each \acrshort{ecu} onto its unique bus system. 
While this layout amplifies communication complexity between the \acrshort{ecu}s, it provides robust isolation between them. 
However, it also reduces the hops per attack path and the number of \acrshort{ecu}s used for the path. 
This design may not be feasible in a real-world scenario due to its complexity but is intriguing to examine its performance. 
It is not expected to do well in the survey.

\section{Architecture 7}
\label{subsec:arch7}

\nameref{fig:architecture7} carries a concept similar to \nameref{fig:architecture6}. 
Instead of placing all \acrshort{ecu}s on their own bus system, it groups \acrshort{ecu}s that use the same bus system.
This configuration amplifies the significance of the bus rating to the overall architecture rating, 
indicating whether the employed bus systems are sufficiently secure. 
Comparing this architecture's performance to Architecture 6's will yield interesting insights.
\section{Architecture 8}
\label{subsec:arch8}

Figure \nameref{fig:architecture8} designates the \acrshort{cgw} as the sole entry point. 
This setup eliminates other entry points, meaning all attack paths originate from the same point, thereby reducing the attack surface. 
Also, the feasibility rating of the interfaces becomes more significant, 
and the attack paths might become much more linear, underscoring the importance of component feasibility ratings. 
This architecture is expected to perform well in terms of security due to the singular entry point. 
However, the entry point's central location may slightly undermine the architecture's security.

\section{Architecture 9}
\label{subsec:arch9}

\nameref{fig:architecture9} is the same as \nameref{fig:architecture4} but excludes a \acrshort{cgw}. 
Again, the dispersed entry points lead to a larger attack surface, and the significance of the \acrshort{cgw} 
can be more accurately deduced from this comparison. 
This architecture is not expected to perform well in security, possibly scoring worse than \nameref{fig:architecture4}.
\section{Architecture 10}
\label{subsec:arch10}

\nameref{fig:architecture10}, the final architecture of the training set, includes only one of each interface, 
but they are more dispersed than in \nameref{fig:architecture8}. 
However, the entry points are still grouped together, aligning with the concept of \nameref{fig:architecture3}. 
This architecture is expected to be one of the most secure due to its few, isolated entry points.

\section{Survey}
\label{sec:survey}

Mercedes-Benz Tech Innovation's security experts were tasked with evaluating ten architectures, 
as detailed in \nameref{chp:arch}. The architects ranked each architecture and provided an explanation 
for their ranking, along with any additional comments. 
The rankings ranged from 1 to 10, with 1 being the most secure and 10 being the least secure. 
The sum of each architecture's rank score determined its final score, 
with the lowest score indicating the most secure architecture.

While there are generally agreed upon factors that make an architecture secure, 
different security architects may approach an evaluation differently based on their background and experience. 
As a result, some factors may be seen as more or less important than others. 
Therefore, the opinion on the security of an architecture can vary between architects. 
The result of this evaluation reflects the mean opinion of the participating architects.\\

The number of hops between the entry and target was identified by the experts as a 
crucial criterion in evaluating the security of attack paths, although not all hops have the same impact on security. 
More hops can make it more difficult for attackers to navigate from the entry point to the target, 
but a shorter but more challenging attack path might be less feasible than a longer but easier one. 

Separation and isolation of domains and ECUs are critical considerations, 
as they enable compartmentalization and application of security measures to each domain or ECU individually. 
The placement of the Powertrain domain in the network is particularly critical, 
given its control of the engine, transmission, and other critical systems. 
The architecture with the external interfaces placed in the same domain (\nameref{fig:architecture3}) 
received the highest score, as expected. 
Similarly, \nameref{fig:architecture10} was expected to score high, as it is similar to \nameref{fig:architecture3} 
regarding the placement of external interfaces, but concerns were raised regarding the Telematic domain controller being a target.

However, isolation has its limits.
Architectures \nameref{fig:architecture6} and \nameref{fig:architecture7} provide high levels of isolation, but at the cost of increased overall complexity.
Communication between \acrshort{ecu}s must also be considered, as it is an essential part of the vehicle. 
Too many isolated \acrshort{ecu}s communicating with each other can result in significant overhead. 
Media change, such as from \acrshort{can} to \acrshort{canfd}, \acrshort{flexray}, or Ethernet, also introduces additional hurdles.
That is why it is astonishing to see that \nameref{fig:architecture6} received such a good score in the survey
and in addition for it to be in such stark contrast to \nameref{fig:architecture7}, which received one of the worst ratings.\\

Other criteria that need to be evaluated include the presence of a \acrshort{cgw} and the number of external interfaces. More interfaces mean more attack vectors and the need for more security mechanisms.
\nameref{fig:architecture4} and \nameref{fig:architecture9} have the most spread out interfaces. Thus they were expected to receive a low score,
whereas Architectures \nameref{fig:architecture8} was expected to receive a high score, as the \acrshort{cgw} is the only entry vector into the network.
However, a \acrshort{cgw} with no external interfaces is overall likely more secure than an architecture with no \acrshort{cgw} at all, and that is why 
\nameref{fig:architecture2} received a higher score than \nameref{fig:architecture1}. 
It is important to consider the type and security of the external interface as each interface has distinct security properties.

While the attack feasibility of each component accounts for the security mechanisms implemented in the vehicle, such as firewalls or IDCs, 
these mechanisms and the component's feasibility have not been explicitly stated in the architectures under consideration since every component can vary in their implementation.
However, these assessments are represented through the feasibility of each \acrshort{ecu}, bus, and interface as explained in Section \nameref{sec:config}.
The experts agreed that the presence of security mechanisms is a critical factor in evaluating the security of the architectures.
Such security measures impede attackers from progressing quickly through the attack path. 
For example, intelligent domain controllers that differentiate between domains are more effective than those that only forward messages.
As a result, the final results varied between experts, as expected.\\

In conclusion, architectures \nameref{fig:architecture3}, and \nameref{fig:architecture8} received the best feedback, 
closely followed by architectures \nameref{fig:architecture6} and \nameref{fig:architecture10} and \nameref{fig:architecture2},
whereas architectures \nameref{fig:architecture4}, \nameref{fig:architecture7}, 
and \nameref{fig:architecture9} received the worst feedback, as they have been ranked last place at least once.

\section{Results - Training Set Architectures}

The result is as follows:

\begin{table}[h]
    \label{table:survey}
    \centering
    \caption{Rank and Architecture}
    \begin{tabular}{ |c|c|c| } 
    \hline
    Rank & Architecture & Score \\
    \hline
    1 & \nameref{fig:architecture3} & 10 \\
    2 & \nameref{fig:architecture8} & 11 \\
    3 & \nameref{fig:architecture6} & 15 \\
    4 & \nameref{fig:architecture10} & 16 \\
    5 & \nameref{fig:architecture2} & 17 \\
    6 & \nameref{fig:architecture1} & 25 \\
    7 & \nameref{fig:architecture5} & 27 \\
    8 & \nameref{fig:architecture7} & 28 \\
    9 & \nameref{fig:architecture9} & 29 \\
    10 & \nameref{fig:architecture4} & 30\\
    \hline
    \end{tabular}
\end{table}
