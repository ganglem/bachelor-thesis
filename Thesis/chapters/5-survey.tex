\chapter{Survey and criteria calibration}
\label{chp:introcriteria}

One of the most important asepcts of this thesis is the criteria used to evaluate the different architectures.

In order to find and calibrate a good criteria, a survey was conducted in which ten architectures were evaluated by security experts in the field of automotive security.
Their rating and feedback on this "training set" was used to calibrate the criteria used on the main architectures in this thesis.
Doing a survey with such a set rather than the architectures themselves was done to avoid biasing the results, as well as have a well evaluated criteria.
In addition, the training set is much larger than the main set, thus the results are more accurate.

\section{Training set architectures}
\label{sec:trainingarch}

Ten architectures were used as a \textit{training set} to determine and calibrate the criteria.
The following section will describe the test architectures and estimate the impact the architectural choices have on the security of the system.

\section{Architecture 1}
\label{subsec:arch1}

Architecture \ref{fig:architecture1} represents the most realistic architecture, 
as it is modeled after an actual architecture used in the automotive industry. 
It offers six entry points with three being targets, one of which is the \textit{Central Gateway}, and eight targets in total. 
It was important to include this architecture, since it could offer valuable insight into the real world.

\section{Architecture 2}
\label{subsec:arch2}

Architecture \ref{fig:architecture2} is essentially the same as Architecture 1, 
but without a central gateway, to see how the architecture would perform without it.

\section{Architecture 3}
\label{subsec:arch3}

Architecture \ref{fig:architecture3} was designed to group all the entry interfaces to see how a centralized entry would affect the architecture. 
Isolating the possible entry locations reduces the attack surface, prolongs the attack path to each target, 
and increases the number of ECUs used for the path, thus making the attack more difficult as the attacker would have to compromise more ECUs. 
It is similar to Architecture 10, in a sense that the entry points are centralized, however, in this case, the entry points are not isolated to a single ECU.

\section{Architecture 4}
\label{subsec:arch4}

Architecture \ref{fig:architecture4} offers entry points from all domain controllers, 
however, notice how every domain controller has only one type of interface. 
Having an entry point from each domain controller enlarges the attack surface, 
as the attacker would only need to compromise one ECU in each domain controller to gain quick access to the whole domain. 
It is essentially the same as Architecture 9, however, including the \textit{Central Gateway}.

\section{Architecture 5}
\label{subsec:arch5}

Architecture \ref{fig:architecture5} uses only Ethernet as the bus system, 
to see how the architecture would perform without any other bus system. 
Ethernet was chosen because it receives the securest attack feasibility rating of any bus system.

\section{Architecture 6}
\label{subsec:arch6}

Architecture \ref{fig:architecture6} puts all the ECUs onto their own bus system. 
This increases complexity in communication between the ECUs but it also offers great isolation between the ECUs, 
however, it also reduces the hops per attack path and number of ECUs used for the path.

\section{Architecture 7}
\label{subsec:arch7}

Architecture \ref{fig:architecture7} has a similar idea to Architecture 6, 
but instead of putting all the ECUs on their own bus system, it puts all the ECUs that used the same bus system on the same bus system. 
Now, the feasibility of the bus rating is much more significant to the overall architecture rating and can indicate whether the used bus systems are secure enough.

\section{Architecture 8}
\label{subsec:arch8}

Architecture \ref{fig:architecture8} only has one entry point, namely the \textit{Central Gateway}. 
This architecture eliminates other entry points meaning the attack paths all start at the same point, thus reducing the attack surface. 
Additionally, the feasibility rating of the interfaces is more significant than before and the attack paths might become much more linear, 
thus the feasibility rating of the components.

\section{Architecture 9}
\label{subsec:arch9}

Architecture \ref{fig:architecture9} is essentially the same as Architecture 4, 
but without a central gateway. Again, the entry points are more dispersed resulting in an enlarged attack surface 
and the significance of the Central Gateway can be deduced more from such a comparison.

\section{Architecture 10}
\label{subsec:arch10}

Architecture \ref{fig:architecture10} is the final architecture of the test set. 
It only includes one of each interface but more dispersed than that of Architecture 8. 
Note that the entries are also grouped together which shares the same idea of Architecture 3.

\section{Survey}
\label{sec:survey}

Security experts at Mercedes-Benz Tech Innovation were given ten architectures as shown in \ref{chp:arch}.
The experts were asked to rank the architectures and give reasoning behind their ranking, as well as add any comments they had.
Since there are ten architectures, the ranking went from 1 to 10, where 1 is the most secure and 10 is the least secure.
To get to a result in the end, each rank was given a score, where the first place got 1 points, the second place got 2 points, and so on.
In the end, each ranking or score each architecture got was summed up, and the architecture with the lowest score was the most secure.\\

Of course there are many ways how a security architect approaches an evaluation.
Each architect comes from a different background with different experience and thus will see some factors as more or less important than another.
Even though architects tend to agree in general about what makes an architecutre secure and what does not, opinions differ in the details.
This result thus represents the architects' mean opinion on the security of the architectures.\\

The experts agreed that the number of hops between the entry and target is a critical criterion in evaluating the security of attack paths. 
The more hops there are, the more challenging it becomes for attackers to navigate from the entry point to the target, however not every hop is the same.\\

Separation and isolation, both of domains and ECUs, is a crucial consideration, as it enables compartmentalization and application of security measures to each domain or ECU individually. 
The Powertrain domain, which controls the engine, transmission, and other critical systems, is of particular concern, and its position in the network must be carefully considered. 
It was thus somewhat expected that Architecture \ref{fig:architecture3} received the highest score, as it has all of the external intefaces placed in the same domain.
Architecture \ref{fig:architecture10} was also expected to receive a high score, as it is similar to Architecture 3 \ref{fig:architecture3} in terms of the placement of the external interfaces,
but the Telematic domain controller being a target was a concern.
Architecture \ref{fig:architecture6} was also not expected to receive a high score, as the the maximum number of hops between the entry and target is 2 on average,
but the experts agreed that isolation played a huge role in the evaluation.\\

However, excessive isolation would make the system too complex. 
Architectures \ref{fig:architecture6} and \ref{fig:architecture7} also provide high levels of isolation, but at the cost of increased overall complexity.
Communication between ECUs must also be considered, as it is an essential part of the vehicle. 
Too many isolated ECUs communicating with each other can result in significant overhead. 
Media change, such as from CAN to CANFD or FlexRay or Ethernet, also introduces additional hurdles.\\

Other factors that need to be evaluated include the presence of a Central Gateway and the number of external interfaces, 
as more interfaces mean more attack vectors and the need for more security mechanisms.
Architecture \ref{fig:architecture4} and Architecture \ref{fig:architecture9} have most spread out interfaces, and thus they were expected to receive a low score,
whereas Architectures \ref{fig:architecture8} was expected to receive a high score, as the Central Gateway is the only entry vector into the network.
However, a Central Gateway with no external interfaces is overall likely more secure than an architecture with no Central Gateway at all, and that is why 
Architecture \ref{fig:architecture2} received a higher score than Architecture \ref{fig:architecture1}. 
Of course, the type and the security of the external interface plays a role as well, because every interface has different security properties.\\

While the attack feasibility of each component accounts for the security mechanisms implemented in the vehicle, such as firewalls or IDCs, 
these mechanisms and the component's feasibility have not been explicitly stated in the architectures under consideration, since every component can vary in their implementation.
However, these feasibility assessments are represented through the feasibility of each ECU, bus, and interface as explained in Section \ref{sec:config}.
The experts agreed that the presence of security mechanisms is a critical factor in evaluating the security of the architectures.
Such security measures impede attackers from progressing quickly through the attack path. 
For example, intelligent domain controllers that differentiate between domains are more effective than those that only forward messages.
As a result, the final results varied between experts, as expected.\\

In general, Architectures \ref{fig:architecture3}, \ref{fig:architecture6}, \ref{fig:architecture8}, and \ref{fig:architecture10} received a positive feedback,
whereas Architectures \ref{fig:architecture4}, \ref{fig:architecture7}, and \ref{fig:architecture9} received a negative feedback.

\section{Results - Training Set Architectures}

The result is as follows:

\begin{table}[h]
    \label{table:survey}
    \centering
    \caption{Rank and Architecture}
    \begin{tabular}{ |c|c|c| } 
    \hline
    Rank & Architecture \\
    \hline
    1 & \ref{fig:architecture3}\\
    2 & \ref{fig:architecture8}\\
    3 & \ref{fig:architecture6}\\
    4 & \ref{fig:architecture10}\\
    5 & \ref{fig:architecture2}\\
    6 & \ref{fig:architecture1}\\
    7 & \ref{fig:architecture5}\\
    8 & \ref{fig:architecture7}\\
    9 & \ref{fig:architecture9}\\
    10 & \ref{fig:architecture4}\\
    \hline
    \end{tabular}
\end{table}
