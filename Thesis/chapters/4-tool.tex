\chapter{Tool}
\label{chp:tool}

In order to evaluate the different complex architecture, a tool was developed.
The tool's purpose is to help security architects quickly evaluate the given vehicular network architecture based on their \gls{attack path} feasibility.\\
Furthermore, the tool is used in this thesis to automatically generate the most feasible attack paths for the different complex architectures, 
as well as evalaute them based on the criteria.

The following sections will describe the tool's implementation.

\section{Configuration Files Structure}
\label{sec:config}

First, the architecture, ECUs and buses need to be defined in a configuration file.
The configuration files for the simulation are stored in JSON format, which is a lightweight data interchange format that is easyto read and write, and easy for machines to parse and generate.

There are three main configuration files used in the simulation: buses.json, ECUs.json, and graph.json.
Each file has a specific structure and contains different attributes.

Note that the feasibility rating is between 0 and 1, where 0 is most feasible, i.e. the attack is most likely to succeed, and 1 is least feasible, i.e. the attack is least likely to succeed.

\subsection{buses.json}
\label{sec:buses}

The tool takes as input three sets of configuration files, each of them represented in the JSON format: 
the system architecture, the ECUs in the system, and the buses connecting the ECUs. 

This file defines the different communication buses that are used in the simulation. 
It contains an array of objects, where each object represents a bus and has the following attributes:

\begin{itemize}
\item \textbf{type}: The type of the bus: \textit{CAN, CANFD, FlexRay, Ethernet, MOST, LIN}.
\item \textbf{feasibility}: A value between 0 and 1 indicating the bus' attack feasibility.
\end{itemize}

\subsection{ECUs.json}
\label{sec:ECUs}

This file defines the electronic control units (ECUs) that are used in the simulation. 

It contains an array of objects, where each object represents an ECU and has the following attributes

\begin{itemize}
\item \textbf{name}: The name of the ECU.
\item \textbf{type}: The type of the ECU (e.g., entry, target, interface).
\item \textbf{feasibility}: A value between 0 and 1 indicating the ECU's attack feasibility.
\end{itemize}

\subsection{graph.json}
\label{sec:graph}

This file defines the topology of the system being simulated, i.e. the architecture. 
It contains an array of objects, where each object represents a communication link which contains the ECUs and interfaces and has the following attributes:

\begin{itemize}
\item \textbf{name}: The name of the link.
\item \textbf{type}: The type of the link (\textit{CAN, CANFD, FlexRay, Ethernet, MOST, LIN, Bluetooth, WiFi, Ethernet, GNSS}).
\item \textbf{ECUs}: An array of the names of the ECUs that are connected by this link.
\end{itemize}

Note that external interfaces (e.g., Bluetooth, WiFi, GNSS) are also considered as ECUs as well as buses in the simulation.
Treating them as such simplifies the implementation of the tool using the following libraries.
The feasibility of the interface can be set in either the buses.json or ecus.json file, however it is recommended to set it in the buses.json file.
Whichever file is used, note that the feasibility must be set to 0 in the other file, to avoid double counting.


\section{Tool Implementation}
\label{sec:implementation}

The tool is implemented in Python 3.10 and newer.
Python was chosen as the implementation language because it is a script-based language, allowing for quick prototyping and development.
Python is also used in the security domain, as well as the proprietary tool used by MBTI.
Additionally, the language offers a vast variety of libraries that can be used to implement the tool.
The most important and useful library in the tool is \textit{networkx} for graph manipulation and generation.\\

\texttt{main()} calls the following functions, which are integral to the tool's functionality:\\

\texttt{parse\_files(architecture, ECU, bus)}: it takes in three parameters: \texttt{architecture, ecu, bus}.
They are the above described configuration files and loads the contents of these files. 
Specifically, the function loads the contents of the architecture file, ECU file and bus file, 
and extracts the architecture name, architecture, ECUs configuration, and buses configuration, respectively. 
The function returns these four objects as a tuple.\\

\texttt{generate\_graph(architecture, ECUs\_config, buses\_config)}:\\
This function generates a directed graph based on the input architecture, ECU configuration, and buses configuration dictionaries. 
It then returns the generated graph along with two lists of entry and target ECUs.
The function first creates empty lists for entry ECUs and target ECUs, and a directed graph \texttt{G} representing 
the vehicular network architecture using the \textit{networkx} library. 
It then creates dictionaries for ECUs configuration and buses configuration using the \textit{name} and \textit{type} attributes respectively.
The function then iterates through each bus in the architecture and extracts its type and feasibility from buses configuration. 
It also extracts the list of ECUs on the bus and for each ECU, it extracts its type and feasibility from ECUs config. 
Based on the ECU type, it adds the ECU to either the entry ECUs or target ECUs list.
Finally, for each ECU on the bus, it iterates through all other ECUs on the same bus and checks their types. 
If the target ECU has the \textit{interface} type and at least one of the ECUs in the current bus is an entry ECU, 
then an edge is added to the graph with a weight equal to the bus feasibility. 
Otherwise, an edge is added between the two ECUs with a weight equal to the sum of the bus feasibility and the target ECU's feasibility.
\textit{Note: This is why it was easier to treat external interfaces as ECUs as well as buses.
Since we are looking for the shortest path for an interface type, not the ECU using it, treating all interfaces
as a bus and ECU conntected to the internal ECUs using such interface facilitates the shortest path generation.}
The function returns the generated graph, along with the entry ECUs and target ECUs lists.\\

\texttt{find\_attack\_path}: To evaluate the graph, the function takes the graph G,
the dictionary representing the entry ECUs, and a list of strings containing the target ECUs. 
It then finds the distance and shortest path from each entry ECU to each target ECU in the graph using the 
\textit{Bellman-Ford algorithm} and returns a dictionary table containing the distance and shortest path for each combination of entry and target ECU.\\
The table dictionary has three keys, \textit{distance}, \textit{shortest\_path}, and \textit{hops} each of which has a value of another dictionary. 
The \textit{distance} dictionary maps each entry ECU to another dictionary that maps each target ECU to its distance from the entry ECU. 
The \textit{shortest\_path} dictionary maps each entry ECU to another dictionary that maps each target ECU to its shortest path from the entry ECU.
\textit{Hops} shows the amount of hops between the entry and target ECUs.\\

In the end, a funciton which applies the criteria (\ref{chp:criteria}) to the table is called, which returns the feasibility rating of the architecture.\\
The evaluation of the architecture is done in a single function and it is easy to extend the tool to support other criteria.
Since the evaluation of the attack feasibility of the architecture is represented by a single number, 
it is easy to rank and compare the results of different architectures.\\

Various \textit{I/O functions} take care of printing the results, saving them to an external file, or quickly saving the graph as an image file
and other \textit{helper functions} are used to simplify the code.