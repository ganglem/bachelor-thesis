%*******************************************************
% Abstract
%*******************************************************

\chapter*{Abstract}\label{chp:abstract}

The increased use of technology in modern vehicles has made cybersecurity a crucial part of the development process of modern vehicles,
making it more and more critical for the safety and security of the vehicle and the privacy and personal data of drivers and passengers.

The internal vehicular network of such E/E systems plays a vital role in the overall cybersecurity of a modern vehicle.
For example, attackers might exploit potential attack paths in the network to gain unauthorized access to a vehicle's systems or networks.

However, most security testing is done at later stages of development, when it is more complex and costly to address vulnerabilities.
Additionally, due to the nature of security testing, it is carried out manually by experts with a high level of expertise and experience with other cybersecurity tools and techniques.
These factors result in a stagnant security testing process that cannot maintain pace with the increasing complexity of modern vehicles.
\\

This thesis focuses on comparing various vehicular network architectures in terms of which offers more security based on their attack path feasibility.
A comprehensive evaluation of multiple architectures is conducted by introducing a tool that integrates the generation of attack paths with an evaluation of each network architecture. 
The results offer a guiding beacon for future development and evaluation of vehicular network designs.