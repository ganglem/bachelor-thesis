%*******************************************************
% Abstract
%*******************************************************

\chapter*{Abstract}\label{chp:abstract}

The increased use of technology in modern vehicles has made cybersecurity a crucial part of the development process of modern vehicles.
Cybersecurity plays an increased role in the safety and security of the vehicle and the privacy and personal data of drivers and passengers.

Most security testing, such as pentesting, is done at later stages of development, a point in time at which it is more complex and costly to address vulnerabilities.
Additionally, due to the nature of security testing, it is carried out manually by experts.
These factors result in a stagnant security testing process that cannot maintain pace with the increasing complexity of modern vehicles.

The internal vehicular network of such E/E systems plays a vital role in the overall cybersecurity of a modern vehicle.
Attackers might exploit potential attack paths in the network to gain unauthorized access to a vehicle's systems or networks.\\

In this thesis, we propose a solution by presenting a tool that can automate the evaluation of vehicular network security based on attack path feasibility.
This tool can automatically generate attack paths in a given vehicular network, find the most feasible ones and evaluate the network's overall security.
The tool will be used to evaluate multiple different vehicular networks and compare their security in terms of attack path feasibility.