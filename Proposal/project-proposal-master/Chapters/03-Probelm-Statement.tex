
%************************************************
\chapter{Problem Statement}
\label{sec:problem}
%************************************************

\section{Project Type}\label{sec:type}

My bachelor thesis segues into the problems described in \ref{sec:field} by proposing the idea for a tool which can be used to automize the evaluation of the security of vehicular networks.
This bachelor thesis is done in cooperation with Mercedes-Benz Tech Innovation.
To give this project a more scietific background, I will do the following: \\

I will conduct attack path analyses on different internal vehicle network architectures.
Those I will compare based on which provides more security with regard to attack paths.\\

First, I will be creating multiple different architecture diagrams, each with different topologies.
Second, I will write a script-based program, which automizes the evaluation of the different topologies.
Finally, I will decide on a criteria, e.g. ASIL, how to rate the different topologies and compare them with it.\\

The project will be both exploratory and implementory in nature, with a focus on the latter.


\section{Project Idea}
\label{chp:requirements}

\graffito{\begin{shaded}\textbf{just implementation and constructive projects}, remove this chapter if your project is of any other type.\end{shaded}}

\begin{shaded}
\noindent
goal of this project, approach

\medskip
\noindent
give a detailed definition of the initial problem that requires a solution, 
respectively the challenges of transferring or synthesizing a concept into a construction or realization

\medskip
\noindent
define use cases and the deduced requirements for the implementation or construction
\end{shaded}


