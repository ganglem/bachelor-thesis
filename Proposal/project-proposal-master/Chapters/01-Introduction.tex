%************************************************
\chapter{Introduction}
\label{chp:introduction}
%************************************************

\section{Research field: Cybersecurity of vehicular networks}\label{sec:field}

Modern vehicles are becoming increasingly reliant on technology, 
with a wide range of systems and components being connected to the internet and each other. 
This includes everything from infotainment systems and navigation to advanced driver assistance systems and autonomous driving features.\\
ISO 26262 describes these so called "E/E Systems" as systems which consists of electrical and electronic elements
and components such as ECUs, sensors, actuators, connections and communication systems like CAN, Ethernet, Bluetooth, etc.\cite{iso26262}\\
As these technologies become more prevalent, the need for strong cybersecurity measures becomes increasingly important. 
Hackers and cybercriminals are constantly finding new ways to exploit vulnerabilities in these systems, 
which can have serious consequences for both the safety and privacy of drivers.\\
Ensuring the safety and security of connected vehicles is crucial to 
the success of the emerging world of connected and autonomous transportation.\\

The internal vehicular network architecture plays a crucial role in the overall cybersecurity of a modern vehicle. 
This is because it determines how different systems and components within the vehicle are connected and communicate with each other.\\
Attack paths play an imprtant role in vehicle networks and security because they help companies to understand the specific routes 
or methods that a malicious actor might use to attack a vehicle's systems or networks. 
Identifying and understanding attack paths is crucial for protecting the safety and security 
of connected and autonomous vehicles, as it allows companies to anticipate and prepare for potential threats.
For example, by understanding the attack paths that might be used to gain unauthorized access to a vehicle's systems, 
companies can implement appropriate security measures to prevent these attacks from occurring.\\
These might include the use of encryption, authentication protocols, and firewall systems to protect against cyber threats.
A well-designed internal vehicular network architecture can help to minimize the risk of cyberattacks.\\

As the number of systems and components that are connected to the internet and each other increases,
so too does the complexity of the internal vehicular network architecture. 
This can make it more difficult to design and implement effective cybersecurity measures, 
as there are more potential points of vulnerability that need to be addressed.\\
Therefore, it is important for companies developing connected and autonomous vehicles 
to prioritize cybersecurity in the design of their internal vehicular network architecture. 
This can help to ensure the safety and security of the vehicle, 
as well as protect the privacy and personal data of drivers and passengers.\\

Methods for security testing like penetration testing, are often carried out in late stages of develpment,
which can lead to the discovery of vulnerabilities at a time when it is more difficult and costly to address them.\\
Additionally, pentesting, is considered to be a skill-based activity that is still carried out manually.
It requires a high level of expertise and experience with other cybersecurity tools and techniques.\\
A crucial element for security assessment is a TARA, a Threat Analysis and Risk Assessment.
Companies perform a TARA during the development process to identify and prioritize potential risks, a
nd to implement controls or countermeasures to reduce or mitigate those risks to an acceptable level.\\

The increased complexity of modern vehicles and arduous nature of the state-of-the-art security testing methods
make it more unfeasibile for companies to conduct security assessment testing as is done right now.
Thus, a need for an automated tool which can help resovle this issue and couple to the TARA process is apparent.


\section{Background and Motivation}
\label{sec:background}

As a computer science student with a passion for cybersecurity and a focus on cyber security in my studies, 
I joined the CarTT Security team at Mercedes-Benz Tech Innovation (MBTI) a year ago as a working student.
I worked with automotive protocols such as CAN, CANFD, FlexRay, and Ethernet,
used software like CANoe, DTS, as well as proprietary tools for pentesting,
and performed scans and created architecture diagrams of vehicular networks.\\

Through my work there, I got to know the field of vehicular cybersecurity.
As already described, the internal vehicular network architecture plays a crucial role in the overall cybersecurity of a modern vehicle.
MBTI is facing the same challenges as mentioned in \ref{sec:field}, and is looking for a solution to this problem.\\

My supervisor and colleague proposed the topic of automated vehicular network evaluation as a way 
to solve the need for a tool to more efficiently assess the security of these systems.
This approach involves generating attack paths automatically, 
which can be used to simulate and assess the security of a system in a more efficient and comprehensive manner. 
Doing so, it is possible to better identify and mitigate potential vulnerabilities early on in the development process
and ultimately improves the overall flexibility and costs of security testing for MBTI.