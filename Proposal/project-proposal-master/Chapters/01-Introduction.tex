%************************************************
\chapter{Introduction}
\label{chp:introduction}
%************************************************

\section{Research field: Cybersecurity of Vehicular Networks}\label{sec:field}

Modern vehicles are becoming increasingly reliant on technology, with a wide range of systems and components being connected to the internet and each other. 
This includes everything from infotainment systems and navigation to advanced driver assistance systems and autonomous driving features.
ISO 26262 describes these so-called "\gls{e/e} Systems" as systems that consist of electrical and electronic elements and components.
Examples include \gls{ecu}s, sensors, actuators, connections, and communication systems like \gls{can}, Ethernet, and Bluetooth \cite{iso26262}.
As these technologies become more and more important, strong cybersecurity measures are also becoming increasingly important. 
Cybercriminals are constantly finding new ways to exploit vulnerabilities in these systems which can have serious consequences for users, 
which includes their safety, privacy, finances, and operational viability \cite{iso21434}. 
Ensuring the safety and security of connected vehicles is crucial to the success of the emerging world of connected and autonomous transportation.
\\

The internal vehicular network architecture plays a crucial role in the overall cybersecurity of a modern vehicle as it determines how different vehicle systems and components are connected and communicate.
\gls{attack path}s play an important role in vehicle networks and security because they define the specific routes that a malicious actor might use to attack a vehicle's systems or networks. 
A well-designed internal vehicular network architecture can help minimize cyberattack risk.
With the increasing number of systems and components that are connected to the internet and each other, the complexity of the internal vehicular network architecture is also increased. 
This can make it more challenging to design and implement effective cybersecurity measures as more potential points of vulnerability arise that need to be addressed. 
Therefore, companies developing connected and autonomous vehicles need to prioritize cybersecurity in designing their internal vehicular network architecture, 
which can help to ensure the safety and security of the vehicle, as well as protect the privacy and personal data of drivers and passengers.
\\

Security testing is, therefore, a crucial part of the development process.
A \gls{tara}, for example, is performed early during the development process to identify and prioritize potential risks.
This information can be used to guide the design and implementation of controls or countermeasures to reduce or mitigate these risks.
Once the system is developed, e.g. a penetration test is carried out to validate the effectiveness of these controls and to identify any remaining vulnerabilities. 
The results of the pentest can then be incorporated back into the TARA process to update the risk assessment and prioritize future risk mitigation efforts. 
For example, companies can implement appropriate security measures by understanding the attack paths that might be used to gain unauthorized access to a vehicle's systems. 
These include encryption, authentication protocols, and firewall systems, etc. to protect against cyber threats.\\
In this way, the combination of a pentest and TARA provides a complete and iterative approach to security assessment.\\
However, security testing is often carried out in the late stages of development, which can lead to the discovery of vulnerabilities at a time when it is more difficult and costly to address.
Additionally, they are considered to be a skill-based activity that is still carried out manually.
It requires a high level of expertise and experience with other cybersecurity tools and techniques. 
The increased complexity of modern vehicles and the arduous nature of state-of-the-art security testing methods make it more unfeasible for companies to conduct them as is done now.\\
Thus, the need for a more efficient approach to improve overall security testing is evident. 
By automizing the process of evaluating automotive network architectures and their attack paths, 
potential vulnerabilities can be assessed early on in the development process resulting in overall more efficient security testing, improving the flexibility, costs and accelerateing it.


\section{Background and Motivation}
\label{sec:background}

As a computer science student with a passion for cybersecurity and a focus on cybersecurity in my studies, I joined the CarIT Security team at Mercedes-Benz Tech Innovation (MBTI) a year ago as a working student. 
I worked with automotive protocols such as CAN, CAN FD, FlexRay, and Ethernet, used software like CANoe, DTS, and proprietary tools for pentesting, performed scans, and created architecture diagrams of vehicular networks. 
Through my work there, I got to know the field of vehicular cybersecurity.
\\

MBTI is facing the same challenges mentioned in \ref{sec:field} and is searching for a solution to this problem. 
As already described, the internal vehicular network architecture plays a crucial role in the overall cybersecurity of a modern vehicle. 
My supervisor and colleague proposed the topic of automated vehicular network evaluation to address the company's need for a tool that can aid in the security testing process.
By automatically evaluating vehicular network architectures, the same benefits as described in \ref{sec:field} can be achieved.