
%************************************************
\chapter{Problem Statement}
\label{sec:problem}
%************************************************

\section{Project Type}\label{sec:type}

My bachelor thesis segues into the problems described in \ref{sec:field} by proposing the idea of a tool that can be used to automate the evaluation of vehicular network security based on \gls{attack path} feasibility.
This bachelor thesis is done in cooperation with Mercedes-Benz Tech Innovation.\\

I will conduct attack path analyses on different internal vehicle network architectures and then compare based on which provides more security regarding attack paths.\\

First, I will create multiple architecture diagrams with different topologies (A1 \ref{sec:a1}).\\
Second, I will write a script-based program that automizes the evaluation of the different topologies 
(F1 \ref{sec:f1})
(F2 \ref{sec:f2})
(F3 \ref{sec:f3})
(F4 \ref{sec:f4})
(F5 \ref{sec:f5}).\\
Finally, I will decide on a criteria (A2 \ref{sec:a2}), how to rate the different topologies, and compare them to conclude which architecture offers the most security (A3 \ref{sec:a3}).\\

The thesis will be both exploratory and implementory in nature, with a focus on the latter.


\section{Thesis Questions}\label{sec:thesis-questions}

The questions this thesis will answer include:

\begin{itemize}
    \item How can different architectures be rated based on attack paths?
    \item How secure is the given vehicular network architecture?
    \item What architectural approach makes a network safer than others?
    \begin{itemize}
        \item How do small changes in network positioning affect the network security overall?
        \item How do simple and more branched out networks compare in terms of security?
    \end{itemize}
\end{itemize}



