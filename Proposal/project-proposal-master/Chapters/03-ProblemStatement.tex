
%************************************************
\chapter{Problem Statement}
\label{sec:problem}
%************************************************

\section{Project Type}\label{sec:type}

My bachelor thesis segues into the problems described in \ref{sec:field} by proposing the idea for a tool which can be used to automize the evaluation of the security of vehicular networks.
This bachelor thesis is done in cooperation with Mercedes-Benz Tech Innovation.
To give this project a more scientific background, I will do the following: \\

I will conduct attack path analyses on different internal vehicle network architectures.
The architectures then I will compare based on which provides more security with regard to attack paths.\\

First, I will be creating multiple different architecture diagrams, each with different topologies (A1)\ref{sec:a1}.\\
Second, I will write a script-based program, which automizes the evaluation of the different topologies 
(F1 \ref{sec:f1})
(F2 \ref{sec:f2})
(F3 \ref{sec:f3})
(F4 \ref{sec:f4})
(F5)\ref{sec:f5}).
Finally, I will decide on a criteria, how to rate the different topologies and compare them with it (A2)\ref{sec:a2}.\\

The project will be both exploratory and implementory in nature, with a focus on the latter.


\section{Thesis Questions}\label{sec:questions}

Since the thesis is based on a tool needed by MBTI, the main question is:\\

\begin{itemize}
    \item How secure is the given vehicular network architecture?
\end{itemize}

However, other questions might be, but are not limited to:\\

\begin{itemize}
    \item What architectural approach makes a network safer than others?
    \item How do small changes in network positioning affect the network security overall?
    \item How do simple and more branched out networks compare in terms of security?
    \item Is a shorter attack path more vulnerable than longer attack paths?
\end{itemize}



