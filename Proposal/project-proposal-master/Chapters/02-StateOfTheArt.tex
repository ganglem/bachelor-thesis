\chapter{State of the Art}
\label{chp:stateoftheart}

There are various approaches to assessing the security of vehicular networks. Cybersecurity standards and frameworks give guidance and best practices for designing, implementing, and testing the cybersecurity of automotive systems and networks. 
The following standards are mentioned in virtually every piece of literature; thus, they are the most important ones to consider:

\begin{itemize}
\item proposes an execution of functional testing and specifies engineering requirements for cybersecurity risk management regarding the concept, product development, production, operation, maintenance, and decommissioning of electrical and electronic E/E architecture systems in road vehicles, including their components and interfaces \cite{iso21434}.

\item provided a guide on vehicle cybersecurity and was created based on existing practices being implemented or reported in industry, government, and conference papers. The best practices are intended to be flexible, pragmatic, and adaptable in their further application to the vehicle industry and other cyber-physical vehicle systems \cite{sae_j3061}.

\item AUTOSAR is a standard for the development of software for electronic control units (ECUs) in the automotive industry \cite{autosar}.
\end{itemize}

The following frameworks also, are mentioned frequently in literature and offer a base for the development of a tool to assess the security of vehicular networks: 

\begin{itemize}

\item Usually, a TARA (Threat and Risk Assessment) \cite{tara} is performed to identify the threats and vulnerabilities of the system. A TARA typically involves using various tools and techniques, such as risk assessment methods, threat modeling, vulnerability assessments, and security testing.

\item An important framework is HEAVENS, which performs risk assessments of general IT systems and models explicitly built for automotive systems. HEAVENS framework uses threat and impact levels to calculate ris \cite{heavens}.

\item Another framework is the EVITA framework, which essentially performs the same things as HEAVENS, but also considers the potential of attacks to impact the privacy of vehicle passengers, financial losses, and the operational capabilities of the vehicles systems and functions \cite{evita}.\\

\end{itemize}



Many of these approaches aim to standardize the process of assessing the security of vehicular networks. However, most of them are based on manual penetration testing and manual vulnerability assessment today. This is because penetration testing is an experienced-based and explorative skill that is difficult to automate. Further research aims to improve or couple existing approaches like performing a TARA. 

\begin{itemize}

\item F. Sommer et al. introduce the concept of Model-Based Security Testing using an EFSM (Extended Finite State Machine) model in their paper "Model-Based Security Testing of Vehicle Networks" \cite{model_based_testing}.
The Automotive Security Model section describes the E-E Architecture, security measures and further development artifacts to protect vehicles against attacks, and the characteristics of attacks, including violated security property, exploited vulnerability, and attacker privileges. 
The model is based on the EFSM, with model elements of Attacker Privileges and transitions of Vulnerability.
The Proof of Concept section of the paper demonstrates how the model can be used to identify different attack paths, analyze the model for attack paths, and execute the attack paths on a real vehicle. 
The incremental approach allows for the redefinition of attack paths, making it useful at different stages during development.
In the Discussion section, the paper notes that the identification of attack paths is similar to performing a TARA (Threat Assessment and Risk Analysis) and that this approach considers security requirements or measures unlike TARA. 
The potential to couple TARA is also mentioned, as well as the need for prioritization due to the large number of vehicular components. 
The paper concludes that this approach can be useful in identifying attack paths and potential vulnerabilities in automotive systems, thus helping to improve the security of vehicles.

\item J. Dürrwang et al. further describe the concept mentioned in "Model-Based Security Testing of Vehicle Networks" and "Attack Path Generation Based on Attack and Penetration Testing Knowledge" in their paper "Automation in Automotive Security by Using Attacker Privileges" \cite{attacker_privileges}.
It defines several types of privileges that an attacker may seek to gain access to a vehicle's communication system and components, such as "Read/Write", "Execute", "Read", "Write", and "Full Control".
The passage notes that channels that are not protected by security measures can be immediately accessed by an attacker, but interpretation is needed in other cases. 
The example given is an attacker connecting to the vehicle via the OBD (On-Board Diagnostics) port, which is connected to the central gateway via CAN (Controller Area Network), and then using the central gateway to gain access to the internal vehicle network.
It also mentioned that the attacker needs to reach one of these privileges to access further attached communication systems and components. 


\item F. Sommer et al. further propose a model-based approach to automate penetration testing of vehicles by using a database of successful vehicular penetration tests. 
The approach is based on the Model-Based Security Testing of Vehicle Networks with EFSM (Extended Finite State Machine) as the foundation. 
The problem with current security testing methods is that they are carried out in the late stages of development and penetration testing is done manually, 
which is considered difficult to automate due to the high complexity of modern vehicles.
The solution proposed is to automate penetration testing with the help of a database containing successful penetration testing and automatically generating attack paths.
The approach and modeling process involve penetration testing and creating a security model based on E-E Architecture, which takes into account the entities that have an impact on the cyber security of a vehicle, and introducing the concept of attacker privileges.
The attack path generation process involves using an attack database that is built based on EFSM. 
The database describes which vulnerabilities and exploits can be used, including attack taxonomy and classification, attack steps, requirements, restrictions, components, and interfaces. 
The database can also be used to find new attack paths by permuting existing attack steps. 
Additionally, the process of creating the database can be done iteratively over several penetration tests and can be transferred to test scripts.
The discussion section highlights that this approach is similar to TARA and can be coupled with it. 
However, there is a risk that the attack path generated may not be transferable, which can be circumvented by permuting previous attacks. 
The authors also suggest that further testing activities, such as black-box testing, should be carried out. 
Despite its limitations, this approach can be used as a useful tool to automate the process of penetration testing and improve the efficiency of security testing in the automotive industry. \cite{attack_database}.

\item In contrast, D. Zelle et al. introduce a concrete approach that can be used in a TARA called "ThreatSurf" \cite{threat_surf}.
They show the feasibility of their approach using an algorithm for automated generation and rating of attack paths using the attack building blocks and attack feasibility.

\end{itemize}



