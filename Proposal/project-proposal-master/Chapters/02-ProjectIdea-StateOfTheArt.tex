%************************************************
\chapter{Project Idea}
\label{chp:idea}
%************************************************

\begin{shaded}
\noindent
detailed discussion of the project's topic, the initial idea and the motivation for pursuing the project
\end{shaded}

As a computer science student with a passion for cybersecurity and a focus on cyber security in my studies, 
I was excited to join the CarTT Security team at Mercedes-Benz Tech Innovation a year ago as a working student. 
When the opportunity arose to complete my bachelor's thesis at the company, 
I was eager to take on the challenge and contribute to the team's efforts.\\

My supervisor and colleague proposed the topic of automated vehicular network evaluation as a way 
to address the need for a tool to more efficiently assess the security of these systems.

One common issue in the field of information security is that security testing is often carried out in the late stages of development, 
which can lead to the discovery of vulnerabilities at a time when it is more difficult and costly to address them. 
Additionally, traditional penetration testing approaches, 
which rely on manual, experience-based, and explorative techniques, 
are considered difficult to automate due to the high complexity of modern systems.\\

To address these issues, there is a growing interest in model-based security testing and 
automating penetration testing with the help of a database containing successful penetration testing techniques. 
This approach involves generating attack paths automatically, 
which can be used to simulate and assess the security of a system in a more efficient and comprehensive manner. 
By leveraging these advanced techniques, 
it is possible to better identify and mitigate potential vulnerabilities early on in the development process.
Doing so ultimately improves the overall flexibility and costs of security testing.\\


%************************************************
\chapter{State of the Art}
\label{chp:stateoftheart}
%************************************************

\begin{shaded}
\noindent
enumerate and discuss related work and practical solutions that form the state of the art

\medskip

\noindent
argue how this state of the art supports the project and where current solutions have shortcomings this project ventures to overcome
\end{shaded}

There a various approaches to assess the security of vehicular networks.

Cybersecurity standards and frameworks give guidance and best practices for designing, implementing, 
and testing the cybersecurity of automotive systems and networks. 
Examples include the ISO 21434 standard for automotive cybersecurity\cite{iso21434},
the SAE J3061 Cybersecurity Guidebook for Cyber-Physical Vehicle Systems\cite{sae_j3061}, 
and the AUTOSAR (AUTomotive Open System ARchitecture) standard for automotive software architecture\cite{autosar}.

Usually, a TARA (Threat and Risk Assessment)\cite{tara} is performed to identify the threats and vulnerabilities of the system.
TARA typically involves the use of a variety of tools and techniques, such as risk assessment methods, 
threat modeling, vulnerability assessments, and security testing. 
It may also involve the use of specialized software or services to automate or streamline the process.

Many of these approaches aim to standardize the process of assessing the security of vehicular networks.
However, most of them are based on manual penetration testing and manual vulnerability assessment as of today.
This is due to the fact that the complexity of modern vehicular networks makes it difficult to automate the process.\\

Further literature aims to improve or couple to already existing appproaches like performing a TARA.\\
F. Sommer, R. Kriesten, and F. Karg propose a model-based method for security testing of vehicle networks\cite{model_based_testing}
using an EFSM (Extended Finite State Machine). 
The nodes of the EFSM are the attacker privileges and the transitions are the actions or vulnerabilities that can be performed by the attacker.\\
J. Dürrwang, F.Sommer and R. Kriesten describe this concept of using EFSMs in "Automation in Automotive Security by Using
Attacker Privileges"\cite{attacker_privileges}.\\
They further propose a method where both concepts are used in combination with a database 
containing successful vehicular penetration tests is proposed to faciliate and 
automate penetration testing by generating attack paths\cite{attack_database}.\\

Overall, the testing of automotive cybersecurity is a complex and evolving field, 
and it involves the use of a variety of tools and techniques to ensure that automotive systems and networks are secure and reliable.


