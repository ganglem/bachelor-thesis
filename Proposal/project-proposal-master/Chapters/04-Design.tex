%************************************************
\chapter{Tool Design}
\label{chp:design}
%************************************************

\section{purpose}\label{chp:purpose}

The purpose of the tool is to help security architects at MBTI to quickly evaluate 
the security of a given vehicular network architecture.

\section{(Functional) Requirements}\label{funct-req}

(Funcitonal) Requirements include but are not limited to:\\

General:

\begin{itemize}
	\item A1\label{sec:a1} Multiple different vehicular network archtitecure diagrams as ARXML files.
	\item A2\label{sec:a2} A criteria to evalute the architectures.
\end{itemize}

Tool:

\begin{itemize}

	\item F1\label{sec:f1}: It will take an ARXML file as input. 
		This ARXML file is an XML file containing the network diagram of the vehicle, which I have created beforehand
		The network diagram consists of ECUs (nodes) and bus systems (edges) connecting the ECUs.
		The ARXML will be parsed to a convenient datatype.
		Fortunately, a member of the MBTI team has already implemented a convenient ARXML parser, that might be used in this tool.

\item F2\label{sec:f2}: Each ECU and each bus system will have a rating, or "difficulty", that can be changed in the script.
\item F3\label{sec:f3}: ECUs can then be marked as entry vectors or target vectors.

\item F4\label{sec:f4}: Next, an algorithm (potentially as mentioned in ThreatSurf\cite{threat_surf}) will find the most feasibile attack path from each entry to each target, 
		based on the ratings.
		However, the algorithm can also be changed in the script.

\item F5\label{sec:f5}: The results are then output to a table, 
		where the overall security of the network architecture is evaluted based on a criteria.

\end{itemize}

\section{Other Requirements}
\label{chp:hardware-software}

other requirements such as hardware, software or non-funcitonal requirements include, but are not limited to:

\begin{itemize}
	\item N1\label{sec:n1}: Tool must be script based
	\item N2\label{sec:n2}: Programming language is Python (version not yet specified)
	\item N3\label{sec:n3}: A virtual machine with a Linux distribution (distribution not yet specified)
	\item N4\label{sec:n4}: Appropriate libraries for the tool, like the ARXML parser from MBTI
	\item N5\label{sec:n5}: An IDE like PyCharm (recommended) or a text editor to change the script
	\item N6\label{sec:n6}: A Git repository where the code and thesis will be stored
\end{itemize}
