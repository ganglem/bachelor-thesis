%************************************************
\chapter{Tool Design and Implementation}
\label{chp:tool}
%************************************************

\section{Purpose}\label{sec:purpose}

The tool's purpose is to help security architects quickly evaluate thegiven vehicular network architecture based on their \gls{attack path} feasibility.

\section{(Functional) Requirements}\label{sec:funct-req}

(Functional) Requirements include but are not limited to:\\

General:

\begin{itemize}
	\item A1\label{sec:a1} Multiple different vehicular network architecture diagrams as files of a certain datatype 
	\item A2\label{sec:a2} A criteria to evaluate the architectures and a survey to calibrate the criteria
	\item A3\label{sec:a3} Compare architectures with one another and conclude which architecture offers the most security
\end{itemize}

Tool:

\begin{itemize}

\item F1\label{sec:f1}: The tool will take files as input and parse them to a convenient datatype: 
One file contains the network diagram of the vehicle.
The network diagram consists of \gls{ecu}s (nodes), bus systems (edges) connecting the ECUs, and interfaces. 
In a separate configuration file, each ECU and each bus system will have an attack feasibility rating. 
It also specifies the entry points (ECUs and interfaces) and targets (ECUs).

\item F2\label{sec:f2}: A graph is created with the parsed data. The ratings in the configuration files are applied to the graph.

\item F3\label{sec:f3}: Next, an algorithm will find the most feasible attack path from each entry to each target based on the ratings. The algorithm can also be changed in the script. 

\item F4\label{sec:f4}: The results are then output to a table containing the feasibility of each entry point to each target point together with the most feasible attack path.

\item F5\label{sec:f5}: Based on the table, the overall security of the network architecture is evaluated using the criteria (see \ref{sec:type} and \hyperref[sec:a2]{A2}).

\end{itemize}

\section{Other Requirements}
\label{sec:hardware-software}

Other requirements such as hardware, software, or non-functional requirements include, but are not limited to:

\begin{itemize}
	\item N1\label{sec:n1}: The tool must be script based
	\item N2\label{sec:n2}: The programming language is Python (version not yet specified)
	\item N3\label{sec:n3}: A virtual machine with a Linux distribution (distribution not yet specified)
	\item N4\label{sec:n4}: Appropriate libraries for the tool
	\item N5\label{sec:n5}: An IDE like PyCharm (recommended) or a text editor to change the script
	\item N6\label{sec:n6}: A Git repository where the code and thesis will be stored
\end{itemize}
